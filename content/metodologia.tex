     \subsection{Linearização do modelo em espaço de estados}
        
    O sistema proposto possui um ponto de equilíbrio na origem do sistema formado pelas variáveis de estado:
    \begin{equation*}
        \begin{split}
        x_1=\ \ &0\\
        x_2=\ \ &0\\
        x_3=\ \ &0\\
        x_4=\ \ &0\\
        \end{split}
    \end{equation*}
    Para realizar a linearização em torno do ponto de operação escolhido calcula-se a Jacobiana do sistema de equações dinâmicas não lineares descrito em \ref{eq:massa_mola_nao_linear} através das equações \ref{eq:Jacobian_A} e \ref{eq:Jacobian_B}. Em seguida, calculou-se o valor da Jacobiana no valor específico do ponto de equilibro, conforme as equações \ref{eq:eval_A} e \ref{eq:eval_B}, obtemos as matrizes $A$ e $B$ do sistema linearizado exibidas a seguir:
 
    \begin{equation*} 
    \begin{split}
        \mathbf{A} =
        \begin{bmatrix}
            0 & 1 & 0 & 0 & \\            
            -\frac{k_{s}^{l}}{m_s}&-\frac{b_{s}^{l}}{m_s}&\frac{k_{s}^{l}}{m_s}&\frac{b_{s}^{l}}{m_s} &\\ \  
            0 & 0 & 0 & 1 & \\
            \frac{k_{s}^{l}}{m_u}&\frac{b_{s}^{l}}{m_u}&-\frac{(k_{s}^{l}+k_t)}{m_u}&-\frac{b_{s}^{l}}{m_u} &\\
        \end{bmatrix};
    \end{split}
    \begin{split}
       \mathbf{x} = 
        \begin{bmatrix}
             x_1 &\\
             x_2 &\\
             x_3 &\\
             x_4 &\\
        \end{bmatrix}; 
    \end{split}
    \end{equation*}
    
    \begin{equation} 
    \begin{split}
        \mathbf{B_u} = 
        \begin{bmatrix}
            0 & \\            
            -\frac{1}{m_s}&\\ \\  
            0 & \\
            \frac{1}{m_u}&\\
        \end{bmatrix};
    \end{split}
    \begin{split}
        \mathbf{B_w} = 
        \begin{bmatrix}
            0 & \\            
            0 &\\ \\  
            0 & \\
            \frac{k_t}{m_u}& \\
        \end{bmatrix}
    \end{split}
    \end{equation}
    
    É esperado que o  sistema linearizado se aproxime do sistema não linear enquanto opera na vizinhança do ponto de equilíbrio utilizado para linearização. 
    
        \subsection{Simulação da resposta temporal do modelo linearizado em malha aberta}
    Abaixo, seguem figuras que ilustram a simulação temporal do sistema não linear e linearizado para 3 diferentes condições iniciais a parâmetros do sistema dados a seguir: 
    
    \begin{equation*} 
    \begin{split}
        \begin{bmatrix}
            x_c & \dot{x}_c & x_w & \dot{x}_w & \\
        \end{bmatrix}=
    \end{split}
    \begin{split}
        \begin{bmatrix}
            -0.01& -0.01& 0& 0& \\
             -0.1&  -0.1& 0& 0& \\
             -0.5&  -0.5& 0& 0&\\
        \end{bmatrix}
    \end{split}
    \end{equation*}
    
   \begin{equation} \label{eq:parametros}
        \begin{split} 
        m_s =   & 290     [kg]\\
        m_u =   & 40      [kg]\\
        b^{l}_s =  & 700  [Ns.m^{-1}]\\
        b^{nl}_s =  & 200      [Ns.m^{-1}]\\
        b^{y}_s=  & 400      [Ns.m^{-1}]\\
        k^{l}_s =  & 235.10^2 [N.m^{-1}]\\
        k^{nl}_s = & 235.10^4 [N.m^{-1}]\\
        k_t =  & 190.10^3 [N.m^{-1}]\\
        \end{split}
    \end{equation}

  

        \subsection{Análise da estabilidade do sistema em malha aberta}
    
    Para o sistema linearizado com parâmetros dados pela equação \ref{eq:parametros} obtemos os seguinte conjunto de autovalores para a Matriz $A$ do sistema:
    
    \begin{equation}
    \begin{split}
    \mathbf{A} =
        \begin{bmatrix}
               0&      1&        0&     0&\\ \\
               -\frac{2350}{29}& -\frac{70}{9}&  \frac{2350}{29}& \frac{70}{29}&\\ \\
               0&      0&        0&     1&\\ \\
                \frac{1175}{2}&   \frac{35}{2}& -\frac{10675}{2}& -\frac{35}{2}&\\ 
        \end{bmatrix}; \ \
    \end{split}
   \begin{split}
   \mathbf{B} =
        \begin{bmatrix}
               0&      0&\\ \\
               0& -\frac{1}{290}&\\ \\
               0&      0&\\ \\
            4750&   \frac{1}{40}&\\
        \end{bmatrix}
   \end{split}     
    \end{equation}
    
    \begin{equation} \label{eq:autovalores}
        \begin{split}
             \lambda=\mathbf{eig(A)}=\
        \end{split}
        \begin{bmatrix}
            -9.0004& +72.3231i&\\
            -9.0004& -72.3231i&\\
            -0.9565& + 8.4588i&\\
            -0.9565& - 8.4588i&\\
        \end{bmatrix}
    \end{equation}
        
    Pode-se observar em \ref {eq:autovalores} que o sistema possui, como autovalores, dois pares de autovalores complexos conjugados com parte real negativa. Logo, pode-se concluir que o sistema é estável internamente. Como estabilidade interna implica em BIBO estabilidade, pode-se dizer que o sistema também é BIBO estável.
    Na figura abaixo podemos observar a localização dos autovalores em malha aberta do sistema linearizado no plano imaginário.
    
    Através do gráfico da figura , pode-se observar que o sistema possui um par de autovalores dominantes, com frequência de oscilação da resposta de $8.4588 rad.s^{-1}$ e taxa de amortecimento de $0.1124$. Além disso, o sistema possui um outro par de autovalores não dominantes com frequência de oscilação da resposta de $72.3231 rad.s^{-1}$ e taxa de decaimento de $0.1235$
    
    \subsection{Análise da Controlabilidade e da observabilidade do sistema linearizado}
    
    Primeiramente definiremos a matriz $\mathbf{C}$ para termos como saída apenas os estados do sistema referentes a posição das massas suspensa e não suspensa, como demonstrado na equação \ref{eq:matriz_c} exibida a seguir:
     \FloatBarrier
    \begin{equation} \label{eq:matriz_c}
        \begin{split}
             \mathbf{C}=
        \end{split}
        \begin{bmatrix}
            1&0&0&0&\\
            0&0&1&0&\\
        \end{bmatrix}
    \end{equation}
    
    As matrizes de controlabilidade e observabilidade são definidas em \ref{eq:matriz_controlabilidade} e \ref{eq:matriz_observabilidade} exibidas a seguir:
    \begin{equation} \label{eq:matriz_controlabilidade}
        \begin{split}
             \mathbf{M_C}=
        \end{split}
        \begin{bmatrix}
            B& AB& A^2B& \cdots& A^{n-1}B&
        \end{bmatrix}
    \end{equation}

    \begin{equation} \label{eq:matriz_observabilidade}
        \begin{split}
             \mathbf{M_O}=
        \end{split}
        \begin{bmatrix}
            C&\\
            CA&\\
            CA^2&\\
            \vdots&\\
            CA^{n-1}&
        \end{bmatrix}
    \end{equation}

    O sistema definido pelas m,atrizes $A$, $B$ e $C$ é controlável se a matriz de controlabilidade $M_C$ tiver posto igual a $n=4$. O sistema será igualmente observável se a matriz de observabilidade $M_O$ tiver posto igual a $n=4$. Para o sistema definido pelo conjunto de parâmetros exibidos em \ref{eq:parametros} obtemos as matrizes $M_C$ e $M_O$ exibidas a seguir:
    \FloatBarrier
    \begin{equation} \label{eq:MC}
          \begin{split}
             \mathbf{M_C}=
        \end{split}
    \begin{smallmatrix}
                   0& -\frac{1}{290}&     \frac{231}{3364}&  \frac{679}{724}&\\  
      -\frac{1}{290}&  \frac{231}{3364}&  \frac{679}{724}&   -\frac{33366}{95}&\\
                   0&  \frac{1}{40}&     -\frac{231}{464}&  -\frac{11425}{91}&\\  
        \frac{1}{40}& -\frac{231}{464}&  -\frac{11425}{91}&  \frac{44200}{9}&\\
    \end{smallmatrix}
    \end{equation}
 
    \begin{equation} \label{eq:MO}
        \begin{split}
             \mathbf{M_O}=
        \end{split}
        \begin{smallmatrix}
           1&              0&              0&              0&\\
           0&              0&              1&              0&\\       
           0&              1&              0&              0&\\       
           0&              0&              0&              1&\\    
       -\frac{2350}{29}&   -\frac{70}{29}&      \frac{2350}{29}&   \frac{ 70}{29}&\\    
        \frac{1175}{2}&     \frac{35}{2}&      -\frac{10675}{2}&  -\frac{35}{2}&\\    
        \frac{43570}{27}&  -\frac{27725}{841}& -\frac{117713}{9}&  \frac{27725}{841}&\\
       -\frac{198889}{17}&  \frac{27725}{116}&  \frac{284473}{3}& -\frac{578725}{116}&\\
        \end{smallmatrix}
    \end{equation}

Dado que a ordem do sistema linearizado é $\mathbf{n}=4$, e como o $Rank(M_C)=4$ e $Rank(M_O)=4$, pode-se afirmar que o sistema linearizado é controlável e observável.

    
        \subsection{Projeto de um controlador por realimentação de estados} \label{sc:projeto_controlador_full}

Foi definida a escolha dos requisitos temporais para projeto do controlador por realimentação de estados separadamente para cada par de autovalores do sistema. Para o par de autovalores de resposta rápida, correspondentes à dinâmica da massa suspensa no eixo das rodas, deseja-se um valor de Máximo overshoot $M_O \leq 10$ \% e Tempo de acomodação para a faixa de tolerância de 2 \% $T_S \leq 0.1$ s. Os resultados numéricos são apresentados com arredondamento na quarta casa decimal.

Dada a equação   para o cálculo da taxa de amortecimento para um valor de máximo overshoot dado, obtém-se o seguinte valor objetivo para $\xi$:

\begin{equation*}
    \xi=-\frac{ln\left(0.1\right)}{\sqrt{\pi^2+ln^2(0.1)}}=0.5912
\end{equation*}

Dada a equação   para o cálculo da $\omega_n$ objetivo para um tempo de assentamento dado, obtém-se o seguinte valor objetivo para $\omega_n$:

\begin{equation*}
    \omega_n=-\frac{ln\left( 0.02*\sqrt{1-0.5912} \right)}{0.1*0.5912}=73.7409
\end{equation*}

A alocação dos autovalores $\lambda = \sigma \pm \omega_n.i$ é determinada a partir da aplicação dos valores encontrados nas equações   e 

\begin{align} \label{eq:autovalores_de_xi}
     \lambda = \omega_n * (-\xi \pm \sqrt{1-\xi^2}.i)
\end{align}

Para os autovalores de resposta rápida obtemos:

\begin{align*} \label{eq:autovalores_nao_dominantes}
     \lambda_1 = -43.5923 + 47.1507*i\\
     \lambda_2 = -43.5923 - 47.1507*i\\
\end{align*}

Para os autovalores de resposta lenta, correspondentes a massa não suspensa do quarto de veículo, deseja-se um valor de Máximo overshoot $M_O \leq 10$ \% e Tempo de acomodação para a faixa de tolerância de 2 \% $T_S \leq 0.5$ s. Os resultados numéricos são apresentados com arredondamento na quarta casa decimal.

Calcula-se os coeficientes $\xi$ e $\omega_n$ da mesma forma que para os autovalores não dominantes assim obtendo os autovalores correspondentes:

\begin{equation*}
    \xi=-\frac{ln\left(0.1\right)}{\sqrt{\pi^2+ln^2(0.1)}}=0.5912
\end{equation*}

\begin{equation*}
    \omega_n=-\frac{ln\left( 0.02*\sqrt{1-0.5912} \right)}{0.5*0.5912}=14.7482
\end{equation*}

\begin{align*} \label{eq:autovalores_dominantes}
     \lambda_3 = -8.7185 + 9.4301*i\\
     \lambda_4 = -8.7185 - 9.4301*i\\
\end{align*}

Desta maneira obtemos a seguinte disposição de autovalores para o sistema em malha fechada:

\begin{equation} \label{eq:autovalores_malha_fechada}
        \begin{split}
             \lambda=\mathbf{eig(A-B_u*K)}=
        \end{split}
        \begin{bmatrix}
            -43.5923& + 47.1507i&\\
            -43.5923& - 47.1507i&\\
            -8.7185& + 9.4301i&\\
            -8.7185& - 9.4301i&\\
        \end{bmatrix}
    \end{equation}

Uma comparação entre os autovalores de malha aberta e malha fechada é exibida na figura exibida a seguir.
    
    O ganho de realimentação para alocação de autovalores em malha fechada é computado através da equação de Lyapunov conforme descrito na seção   conforme exibido a seguir:
    
    \begin{equation*} 
    \begin{split}
        \mathbf{F} =
        \begin{bmatrix}
            -43.5923 &  47.1507 &       0       & 0 & \\ 
            -47.1507 & -43.5923 &       0 &       0 & \\
                   0 &        0 & -8.7185 &  9.4301 & \\
                   0 &        0 & -9.4301 & -8.7185 & \\
        \end{bmatrix}
    \end{split}
    \end{equation*} 
 
    \begin{equation*} 
    \begin{split}
        \mathbf{B} = B_u = 
        \begin{bmatrix}
            0 & \\
            -\frac{1}{m_s}&\\ \\
            0 & \\
            \frac{1}{m_u} \\
        \end{bmatrix};\ \
    \end{split}
    \begin{split}
    \mathbf{\bar{K}} =
        \begin{bmatrix}
        1&0&1&0&
        \end{bmatrix}
    \end{split}
    \end{equation*} 
    
    Resolvendo a equação de Lyapunov \ref{eq:lyapunov} para $T$ e substituindo a matriz T encontrada na equação \ref{eq:ganho} obtém-se os seguintes valores de ganhos de realimentação, valores arredondados na quarta casa decimal:
    
    \begin{center}
    \begin{tabular}{|c|c|}
        \hline
        Estado & Ganho\\
        \hline
        \hline
        $x_1$    & -18023.2837\\
        $x_2$    & -4567.6793\\
        $x_3$    & 13118.63414\\   
        $x_4$    & 2758.2869 \\
        \hline
    \end{tabular}
    \end{center}
    
        \subsection{Simulação da resposta temporal do modelo linearizado em malha fechada com controlador por realimentação de estados} \label{sc:analise_resposta}
    
    Abaixo seguem gráficos que ilustram a simulação temporal da resposta do sistema linearizado para excitação com entrada em "lombada" degrau de 0.1m:
    
    A observação dos gráficos da resposta temporal exibida nas figuras   traz algumas considerações, enumerados a seguir:
    
    \begin{itemize}
        \item A realimentação de estados, na forma em que foi projetada, preservou a estabilidade do sistema.
        \item A realimentação de estados sozinha foi capaz de alterar os autovalores, ou autovalores, do sistema para as localizações desejadas.
        \item A nova localização dos autovalores foi capaz de alterar as frequências de resposta forçada do sistema. Isso contribuiu para que este atingisse a estabilização da resposta transitória mais rapidamente do que quando em malha aberta, atingindo as especificações do tempo de assentamento.
        \item A nova localização dos autovalores foi capaz de atenuar mais rapidamente as frequências de resposta forçada do sistema. Isso contribuiu para que este atingisse a estabilização da resposta transitória mais rapidamente do que quando em malha aberta além de ser capaz de reduzir o overshoot.
        \item A realimentação de estados sozinha não foi capaz de zerar o erro do estado $\dot{x}_c$ para referência zero quando o sistema foi excitado por um degrau. O que não aconteceu para o sistema em malha aberta. Considerando que o sistema de coordenadas é relativo ao sistema de coordenadas do solo e ao offset inicial de cada componente, espera-se que em regime permanente os estados $x_c$ e $x_w$ estejam na mesma posição que a excitação em $x_r$.
        \item A realimentação de estados tornou o sistema mais rápido por um curto espaço de tempo durante o tempo de subida. Apesar de estar sujeito a um menor overshoot, as massas do sistema sofreram maiores acelerações de pico durante o transitório.
    \end{itemize}
    
    De uma maneira geral a resposta do sistema controlado é bastante satisfatória.
    
        \subsection{Simulação da resposta temporal do sistema em malha fechada considerando o sistema não linear original utilizando o controlador projetado para o sistema linear}
    
    Abaixo seguem gráficos que ilustram a simulação temporal da resposta do sistema não linear comparada com a resposta do sistema linearizado para excitação com entrada em "lombada" degrau de 0.1m:

    
    A observação dos gráficos da resposta temporal exibida nas figuras  traz a conclusão que, para este caso específico, a resposta temporal do sistema não linear em malha fechada com o controlador projetado para o sistema linearizado se mostrou muito semelhante a resposta temporal do sistema linear. Neste caso, contribuíram para isto a excitação com "lombada" em degrau pequena afastou pouco o sistema do ponto de equilíbrio, ao mesmo tempo que um controlador com um amortecimento relativamente alto, $\xi \simeq 0.6$, contribuiu para que a resposta transitória rica em frequências fosse rapidamente cancelada e apenas o comportamento em regime fosse exibido a maior parte do tempo.
    
    Para comparação, é exibido a seguir a mesma simulação temporal para uma entrada com "lombada" em degrau unitário. Esta entrada excita mais frequências de ambos os sistema e podem revelar mais informações sobre as diferenças entre ambos os sistemas pois leva o sistema linearizado para uma condição de operação mais afastada do ponto de equilíbrio: 
    
    A observação dos gráficos da resposta temporal exibida nas figuras demonstra que para uma entrada que desloque o sistema para uma região muito distante do ponto de equilíbrio utilizado para linearização pode causar uma grande diferença na resposta temporal do controle em malha fechada, inclusive podendo degradar significativamente o desempenho do controlador por realimentação de estados.  
        \subsection{Simulação da resposta temporal em malha fechada com o controlador projetado para o sistema linearizado e considerando perturbação}
 
    Abaixo seguem gráficos que ilustram a simulação temporal da resposta do sistema linearizado para excitação com perturbações em degraus aleatórios entre $\pm$ 5cm.

    Para a análise da resposta dinâmica do sistema controlado permanecem válidas as mesmas considerações apresentadas na seção \ref{sc:analise_resposta}

        \subsection{Projeto de um controlador baseado no observador para o sistema linearizado considerando
        que apenas algumas variáveis de estado não são mensuráveis}

    Considerou-se para efeitos práticos que as variáveis de estado $x_w$ e $\dot{x}_w$, correspondentes a posição vertical do eixo da roda e sua respectiva velocidade, não são mensuráveis.

    Foi decidido que a dinâmica do estimador de estados deve ser 4 vezes  mais rápida do que a dinâmica do sistema linearizado, cujos autovalores são exibidos em \ref{eq:autovalores}. Foi adicionado um valor de offset aos autovalores da planta original, os autovalores resultantes para a dinâmica do preditor linear são exibidos a seguir:

    \begin{equation} \label{eq:autovalores_pred}
        \begin{split}
              \mathbf{eig(A-L*C)}=\
        \end{split}
        \begin{bmatrix}
             -36.0016& +72.3231i&\\
             -36.0016& -72.3231i&\\
             -27.9577& +8.4588i&\\
             -27.9577& -8.4588i&\\
        \end{bmatrix}
    \end{equation}
    Na figura  é demonstrado como ficou o design de todos os autovalores do projeto.

    Os autovalores foram utilizados para definir o conjunto de ganhos $L$ do observador de estados através da fórmula de Lyapunov definida em \ref{eq:lyapunov}. No entanto empregou-se a seguinte substituição de variáveis:
    \begin{equation} \label{eq:var_pred}
        \begin{split}
        A = A^{t}&\\
        B = C^{t}&
        \end{split}
    \end{equation}

    Desta maneira, foi encontrada a seguinte matriz de ganhos de realimentação para o observador de estados linear, números arredondados na quarta casa decimal:

    \begin{equation} \label{eq:ganhos_pred}
        \begin{split}
            \mathbf{L}=\
        \end{split}
        \begin{bmatrix}
          61.7626&   61.7626&   61.7626&   61.7626&\\
         760.6968&  760.6968&  760.6968&  760.6968&\\
          46.2422&   46.2422&   46.24227&  46.2422&\\
        3075.9999& 3075.9999& 3075.9999& 3075.9999&\\
        \end{bmatrix}
    \end{equation}

    Uma simulação temporal do sistema não linear controlado pelo ganho de realimentação de estados, obtido através do preditor desenvolvido nesta seção é exibido nas figuras abaixo:

    Houve uma piora no desempenho do do sistema real controlado em malha fechada quando se aplicou a restrição dos estados observáveis, quando comparado com a situação onde todos os estados são observáveis. Especialmente durante os transitórios da perturbação. No entanto, para pequenas variações, o controle baseado em realimentação de estados de um observador linear se mostrou satisfatório.
    