É possível concluir com este trabalho que um sistema não linear pode ser controlado utilizando técnicas de controle de sistemas lineares, desde que seja possível realizar a construção de mu modelo matemático deste sistema que possa ser linearizado na proximidade de um ponto de equilíbrio por meio da expansão em série de Taylor. É possível concluir que um modelo linearizado em espaço de estados do sistema possui comportamento consistente e semelhante ao sistema original para simulações que se afastem pouco do ponto de equilíbrio escolhido para a linearização. 
Dentro do escopo da análise da estabilidade do sistema autônomo, a linearização não altera a característica de estabilidade do sistema em malha aberta, quando comparado com a característica de estabilidade do sistema original. Desta forma é possível aplicar os critérios de avaliação de estabilidade para o sistema linearizado e os resultados obtidos serão consistentes se comparados com o sistema original. 
Dentro do escopo da análise da controlabilidade e da observabilidade do sistema, a linearização não altera as características de controlabilidade e da observabilidade, quando comparado com o sistema original. Desta forma foi possível aplicar as técnicas de análise de sistemas lineares para projetar um observador de estados linear que se comporta de maneira consistente ao sistema original e que inclusive é capaz de estimar estados não mensuráveis deste por meio da realimentação do erro da estimativa dos estados. Foi possível também sintetizar os ganhos do controlador por realimentação de estados que garantisse, simultaneamente, a estabilização do sistema controlado em malha fechada e o desempenho mínimo requerido a projeto para as especificações da resposta transitória para sistemas subamortecidos.
Dentro do escopo da validade dos modelos lineares empregados, é possível afirmar que todo o sucesso do projeto de controle está intrinsecamente relacionado a distância entre os estados do sistema linear e o ponto de equilíbrio utilizado para linearização. Todos os parâmetros do modelo do sistema linearizado são dependentes do ponto de equilíbrio escolhido, e por consequência, o erro entre a resposta temporal do sistema real não-linear e do modelo linearizado depende desta diferença de maneira muito sensível.