\subsection{Modelo numérico do sistema de suspensão ativa}
Com base nas matrizes A e B do sistema linearizado obtidas em \ref{ed:linsys:ax} e \ref{ed:linsys:BuBw} e nos parâmetros na tabela \ref{tb:parametros} do Apêndice, obtemos os seguintes valores numéricos do sistema LTI que será utilizado no benchmark deste trabalho, considerou-se que o sistema é um LPV com incerteza paramétrica $\delta$ na massa do veiculo, podendo variar em $\pm 100 kg$ ao redor do valor nominal do modelo de dependendo do carregamento do veiculo:
%\ms, \mus, \bls, \bnls, \bys, \kls, \knls, \kt

\begin{equation}\label{ed:linsys:ALPV}
    \begin{split}
        \mathbf{A} =
        \begin{bmatrix}
            0 & 1 & 0 & 0 & \\            
            -\frac{\kls}{\ms+\delta}&-\frac{\bls}{\ms+\delta}&\frac{\kls}{\ms+\delta}&\frac{\bls}{\ms+\delta} &\\ \  
            0 & 0 & 0 & 1 & \\
            \frac{\kls}{\mus}&\frac{\bls}{\mus}&-\frac{(6925.1077\cdot10^{3})}{\mus}&-\frac{\bls}{\mus} &\\
        \end{bmatrix}
    \end{split}
\end{equation}

\begin{equation}\label{ed:linsys:BLPV}
    \begin{split}
        \mathbf{B_u} = 
        \begin{bmatrix}
            0 & \\            
            -\frac{1}{\ms+\delta}&\\ \\  
            0 & \\
            \frac{1}{\mus}&\\
        \end{bmatrix};
    \end{split}
    \begin{split}
        \mathbf{B_w} = 
        \begin{bmatrix}
            0 & \\            
            0 &\\ \\  
            0 & \\
            \frac{\kt}{\mus}& \\
        \end{bmatrix}
    \end{split}
\end{equation}

Para a matriz $\mathbf{C}$ consideraremos ser possível a leitura de todos os estados do sistema, e para as matrizes $D_u$ e $D_w$, consideramos que não há termo de transmissão direta das entradas, como exibido a seguir:

\begin{equation} \label{eq:matriz_c}
    \begin{split}
        \mathbf{C}=
    \end{split}
    \begin{bmatrix}
        1&0&-1&0&\\
        0&1&0&0&\\
        0&0&0&1&\\
        0&0&0&0&\\        
    \end{bmatrix};\ \
    \begin{split}
        \mathbf{D_u}= 
    \end{split};\ \
    \begin{bmatrix}
        0&\\
        0&\\
        0&\\
        0&\\
    \end{bmatrix};\ \
    \begin{split}
        \mathbf{D_w}=
    \end{split};\ \ 
    \begin{bmatrix}
        0&\\
        0&\\
        0&\\
        0&\\
    \end{bmatrix};\ \
\end{equation}

Nesta configuração, a saída $y1$ é o curso da suspensão, a saída $y2$ é a velocidade vertical da massa principal e a saída $y3$ é a velocidade vertical da massa das rodas.

\subsection{Especificação da região de \( \mathcal{D}\)-estabilidade desejada}
Deseja-se um valor de Máximo overshoot percentual de $M_O \leq 10$ \% e Tempo de acomodação para a faixa de tolerância de 2 \% $T_S \leq 0.5$ s. 
Dada a equação para o cálculo da taxa de amortecimento para um valor de máximo overshoot dado, obtém-se o seguinte valor objetivo para $\xi$:

\begin{equation*}
    \xi=-\frac{ln\left(0.1\right)}{\sqrt{\pi^2+ln^2(0.1)}}=0.5912
\end{equation*}

Dada a equação para o cálculo da $\omega_n$ objetivo para um tempo de assentamento dado, obtém-se o seguinte valor objetivo para $\omega_n$:

\begin{equation*}
    \omega_n=-\frac{ln\left( 0.02*\sqrt{1-0.5912} \right)}{0.5*0.5912}=13.5328
\end{equation*}

Para $\xi=0.5912$ e $\omega_n=13.5328$ os parâmetros $\sigma_p$ e $\theta$ são obtidos como se segue

\begin{equation}\label{ed:dstab:params}
    \begin{split}
       \sigma_p=-\omega_n=-13.5328\ \ rad\cdot s^{-1}\\
       \beta_p=-20\cdot\omega_n=-270.6566\ \ rad\cdot s^{-1}\\
       \theta=\pi-\cos{(\xi)}^{-1}=0.9383\ \ rad\\    
    \end{split}
\end{equation}

Apenas para limitar a região de \( \mathcal{D}\)-estabilidade desejada, tomaremos o limite a esquerda como sendo $20\cdot\sigma_p$. A região de \( \mathcal{D}\)-estabilidade definida para acomodação do sistema é exibida na figura \eqref{eq:dsta:regiaodef} a seguir:

\FloatBarrier
\begin{figure}[htbp] 
  \begin{centering}
    \includegraphics[width=7.5cm]{img/regiao_d_estabilidade.png}
    \caption{Região definida de estabilidade}
    \label{eq:dsta:regiaodef}
  \end{centering}
\end{figure}
\FloatBarrier

\subsection{Controlador robusto com \( \mathcal{D}\)-estabilidade garantida}\label{subsec:res:control}
Considerando o sistema no espaço de estados apresentado em \eqref{eq:dsta:linsys} com incerteza paramétrica nas matrizes $A$ e $B$, cujos valores são apresentados em \eqref{ed:linsys:ALPV} e \eqref{ed:linsys:BLPV}, em que o parâmetro $\delta$ representa a incerteza do modelo associada a massa do veiculo de $\pm100kg$. A síntese do  controlador  robusto ́e  realizada  pela  resolução das LMIs \eqref{eq:dsta:lmis} e \eqref{eq:dsta:lmisaux} para os parâmetros da região de \( \mathcal{D}\)-estabilidade definidos em \eqref{ed:dstab:params}. Os ganhos obtidos do procedimento de otimização são exibidos abaixo com precisão de 4 casas decimais:

\begin{equation*} \label{eq:ganhoscontrolador}
    K^'=
    \begin{bmatrix}
        -533009.5068&\\	
        -53068.2892&\\	
        808655.8099&\\
        2269.8250&\\
    \end{bmatrix}
\end{equation*}

Calculou-se a localização dos autovalores do sistema para os valores de $\delta = -100, -90, \dots, 90, 100$ e os resultados são exibidos na figura abaixo, com o sistema em malha aberta e malha fechada.
\FloatBarrier
\begin{figure}[htbp]
    \begin{centering}
    \includegraphics[width=7.5cm]{img/regiao_d_estabilidade_autoval.png}
    \includegraphics[width=6cm]{img/regiao_d_estabilidade_autoval_leg.png}
    \caption{Autovalores do sistema}
    \label{fig:regiao_d_estabilidade_autoval_leg}
    \end{centering}
\end{figure}
\FloatBarrier

Ao analisar o gráfico, nota-se que o controlador projetado foi capaz de manter os autovalores responsáveis pela resposta dinâmica do sistema em malha fechada dentro da região de \( \mathcal{D}\)-estabilidade para todos os valores testados do parâmetro $\delta$, sendo então um controlador que garante os requsitos de desempenho temporal.
\subsection{Observador de Luenberger}
Projetou-se a dinâmica do estimador de estados para ser 3 vezes  mais rápida do que a dinâmica do sistema em malha fechada, Adicionou-se o valor de $\beta_p=-270.6566$ como offset aos autovalores da planta original.
Estes foram utilizados para definir o conjunto de ganhos $L$ do observador de estados através da fórmula de Lyapunov definida em \eqref{eq:lyapunov}. 
Desta maneira, foi encontrada a seguinte matriz de ganhos de realimentação para o observador de estados:

\begin{equation} \label{eq:ganhos_pred} \small
    \begin{split}
        \mathbf{L}=\
    \end{split}
    \begin{bmatrix}
       -14726.1365& -14726.1365& -14726.1365& -14726.1365&\\
        21127.9445&  21127.9445&  21127.9445&  21127.9445&\\
        -1416.1998&  -1416.1998&  -1416.1998&  -1416.1998&\\
        -6735.3815&  -6735.3815&  -6735.3815&  -6735.3815&\\
    \end{bmatrix}
\end{equation}
 
\subsection{Observador de entradas desconhecidas}
Aplicando o Teorema \ref{theo:UIO}, as condições LMIs são satisfeitas e obteve-se um limitante $\gamma=1$ e um residuo de $res1.5215\cdot10^{-11}$. As matrizes constantes obtidas para o Observador são exibidas a seguir

\begin{equation*}\label{ed:UIO:Ah}
    \begin{split}
        \mathbf{A_h} =
        \begin{bmatrix}
            -0.9182&     0.1101&   0.0273&  0.0000&\\
            -15.0148& -146.6523& 15.0148&  0.0000&\\
              0.0273&   -0.1101& -0.9182& -0.0000&\\
              0.0000&   0.0000& -0.0000& -0.9455&\\
        \end{bmatrix}
    \end{split}
\end{equation*}

\begin{equation}\label{ed:UIO:L}
    \begin{split}
        \mathbf{L} =
        \begin{bmatrix}
              0.4727& 0.8899& 0.0000& 0&\\
            -76.9555& 143.9127& 2.7395& 0&\\
             -0.4727& 0.1101& 1.0000& 0&\\
             -0.0000& -0.0000& 0.0000& 0&\\
        \end{bmatrix}
    \end{split}
\end{equation}
\begin{equation*}\label{ed:UIO:E_Bh}
    \begin{split}
        \mathbf{E} =
        \begin{bmatrix}
            0& 0& 0& 0&\\
            0& 0& 0& 0&\\
            0& 0& 0& 0&\\
            0& 0&-1& 0&\\
        \end{bmatrix}
    \end{split}
    \begin{split}
        \mathbf{B_h} =
        \begin{bmatrix}
            0&\\
            -0.0053&\\
            0&\\
            0&\\
        \end{bmatrix}
    \end{split}
\end{equation*}

\subsection{Comparação do erro de estimação de estado}

Uma simulação temporal do sistema linear em malha aberta é exibido nas figuras a seguir. A figura \ref{fig:setup_testeobs} no Apêndice mostra o experimento desenvolvido em Simulink para testar os observadores.
Para esta simulação considerou-se o sistema operando na condição nominal com $\delta=0$, em malha aberta, e uma entrada de perturbação como exibida na equação \eqref{eq:estatdos_perturbacao} e na figura \ref{fig:estatdos_perturbacao}.

 Perturbação para teste do observador
\begin{equation}\label{eq:estatdos_perturbacao}  \footnotesize
    \omega(t)=0.6sin(8\pi t)+0.75sin(12\pi t)+0.9sin(16\pi t)+0.5sin(20\pi t)
\end{equation}

\begin{figure}[htbp]
    \begin{centering}
    \includegraphics[width=7cm]{img/estatdos_perturbacao.png} 
    \caption{Perturbação para teste do observador}
    \label{fig:estatdos_perturbacao}
    \end{centering}
\end{figure}

A figura \ref{fig:estatdos_luenberger} mostra a comparação entre os estados lidos e preditos no observador de Luenberger enquanto a figura \ref{fig:estatdos_UIO} mostra a mesma informação para o observador UIO. As figuras \ref{fig:estatdos_erro_luenberg} e \ref{fig:estatdos_erro_UIO} mostram o erro de predição do preditor de Luenberger e UIO respectivamente.

\FloatBarrier
\begin{figure}[htbp]
    \begin{centering}
    \includegraphics[width=7cm]{img/estatdos_luenberger.png} 
    \caption{Comparação dos estados do sistema com o observador de Luenberger}
    \label{fig:estatdos_luenberger}
    \end{centering}
\end{figure}

\begin{figure}[htbp]
    \begin{centering}
    \includegraphics[width=7cm]{img/estatdos_erro_luenberg.png} 
    \caption{Erro do observador de Luenberger}
    \label{fig:estatdos_erro_luenberg}
    \end{centering}
\end{figure}
\FloatBarrier

Simulou-se o sistema em conjunto com o observador aplicando a mesma perturbação definida em \eqref{eq:estatdos_perturbacao}.
A figuras \ref{fig:estatdos_luenberger} a seguir mostram a evolução dos estados reais e preditos enquanto a figura \ref{fig:estatdos_erro_luenberg} mostra o erro de predição dos estados.

\FloatBarrier
\begin{figure}[htbp]
    \begin{centering}
    \includegraphics[width=7.5cm]{img/estatdos_UIO.png} 
    \caption{Comparação dos estados do sistema com o observador de entradas desconhecidas}
    \label{fig:estatdos_UIO}
    \end{centering}
\end{figure}
\FloatBarrier

\FloatBarrier
\begin{figure}[htbp]
    \begin{centering}
    \includegraphics[width=7.5cm]{img/estatdos_Erro_UIO.png} 
    \caption{Erro do observador de entradas desconhecidas}
    \label{fig:estatdos_erro_UIO}
    \end{centering}
\end{figure}
\FloatBarrier

Os dados das simulações temporais exibidos nas figuras \ref{fig:estatdos_erro_luenberg} e \ref{fig:estatdos_erro_UIO} foram compilados em tabelas onde se exibem o erro médio, o erro máximo e o erro RMS.
\FloatBarrier
\begin{table}[h!]
\footnotesize
\centering
    \begin{tabular}{|c|c|c|c|c|}
        \hline
        Luenberger & $X_c$& $V_c$& $X_w$& $V_w$\\
        \hline
        \hline
         Erro Médio&  1.2892&   -7.5121&   -4.6548&   10.9155\\ 
         Erro Máximo& 76.2538&  157.5491&  123.1720&  202.6507\\ 
         Erro RMS&    22.7196&   58.9452&   53.7731&   93.5771\\
        \hline
    \end{tabular} \label{tb:comparacao_erro_estado_LUE}\caption{Características do erro de estimação de estados com observador de Luenberger}
\end{table}

\begin{table}[h!]
\footnotesize
\centering
    \begin{tabular}{|c|c|c|c|c|}
        \hline
        UIO & $X_c$& $V_c$& $X_w$& $V_w$\\
        \hline
        \hline
         Erro Médio&  0.0090&   -0.0198&    0.0089&   -0.0000\\ 
         Erro Máximo& 0.0814&    0.5692&    0.0821&    0.0000\\ 
         Erro RMS&    0.0253&    0.2258&    0.0258&    0.0000\\
        \hline
    \end{tabular} \label{tb:comparacao_erro_estados_UIO}\caption{Características do erro de estimação de estados com observador de Luenberger}
\end{table}
\FloatBarrier

A análise dos resultados exibidos nas tabelas \ref{tb:comparacao_erro_estado_LUE} e \ref{tb:comparacao_erro_estados_UIO} indica que o observador de entradas desconhecidas performou muito melhor do que o observador de Luenberger para todas as variáveis de estado. 
Na análise do valor do erro médio, é esperado que este se aproxime de zero, esta métrica para o observador de Luenberger isso não aconteceu sob perturbação, para o observador UIO o valor do erro médio é próximo de zero com precisão de pelo menos $10\cdot^{-3}$. 
O erro Máximo e o erro RMS tem relação com o erro de estimação durante a resposta transitória do sistema. Ambos são da ordem de $10\cdot^{0}$ para o observador UIO nas variáveis de estado $Xc$ e $Vc$. Enquanto isso, o erro máximo e o erro RMS alcança valores da ordem de $10\cdot^{-4}$ para $Vc$ e $10\cdot^{-9}$ para $Xc$ quando o observador UIO é empregado.
É interessante notar que a ordem de grandeza do erro para as variáveis $Xw$ e $Vw$ são idênticas em ambos os casos. Isso pode ser explicado pela ausência da informação medida dessas variáveis na matriz $C$ do sistema. Ao observar os gráficos destas variáveis de estado em ambos os observadores, nota-se que o valor estimado de ambas é próximo de zero em ambos os observadores.

\subsection{Comparação da estratégia de controle da transmissão de vibrações}
Para avaliar a eficiência das estratégias de controle serão considerados três critérios a velocidade vertical da massa do veiculo, a aceleração e o \emph{Jerk}. Jerk, ou arranco ou arrancada, é definido como a derivada da aceleração e é uma medida de conforto veicular. No caso pretende-se avaliar qual estratégia de controle foi mais eficaz em promover o amortecimento da vibração. Enfatize-se que ambas utilizam o mesmo controlador robusto definido na seção \ref{subsec:res:control}, diferindo apenas no tipo de observador de estados empregado.
Na figura \ref{fig:setup_testectr} no Apêndice é exibido o setup do experimento em Simulink para simulação das estratégias de controle.

Nas figuras a seguir são exibidos a velocidade de deslocamento vertical da massa do veiculo, a aceleração e o \emph{Jerk}. A figura \ref{fig:eval_cntrol_LUE} a seguir corresponde ao sistema de controle com observador de Luenberger. A figura \ref{fig:eval_cntrol_UIO} adiante corresponde ao sistema de controle com observador UIO.
\FloatBarrier
\begin{figure}[htbp]
    \begin{centering}
    \includegraphics[width=7.5cm]{img/eval_cntrol_LUE.png} 
    \caption{Saídas de desempenho do sistema em malha fechada com o controlador robusto e observador de Luenberger}
    \label{fig:eval_cntrol_LUE}
    \end{centering}
\end{figure}
\FloatBarrier

\begin{figure}[htbp]
    \begin{centering}
    \includegraphics[width=7.5cm]{img/eval_cntrol_UIO.png} 
    \caption{Saídas de desempenho do sistema em malha fechada com o controlador robusto e observador de entradas desconhecidas}
    \label{fig:eval_cntrol_UIO}
    \end{centering}
\end{figure}
\FloatBarrier

Os dados das simulações temporais exibidos nas figuras \ref{fig:eval_cntrol_LUE} e \ref{fig:eval_cntrol_UIO} foram compilados em tabelas onde se exibem o valor médio, o valor máximo e o valor RMS para as variáveis de interesse.
\FloatBarrier
\begin{table}[h!]
\footnotesize
\centering
    \begin{tabular}{|c|c|c|c|}
        \hline
        Luenberger& $V_c$& $A_c$& $Jerk_c$\\
        \hline
        \hline
         Valor Médio&   -0.1154&    0.0004&    0.0001\\ 
         Valor Máximo&  48.6894&    1.1632&    0.5598\\ 
         Valor RMS&     17.5553&    0.4459&    0.0257\\
        \hline
    \end{tabular} \label{tb:comparacao_controle_LUE}\caption{Características do sistema de controle de vibração com observador de Luenberger}
\end{table}

\begin{table}[h!]
\footnotesize
\centering
    \begin{tabular}{|c|c|c|c|}
        \hline
        UIO& $V_c$& $A_c$& $Jerk_c$\\
        \hline
        \hline
         Valor Médio&   -0.0386&   -0.0005&    0.0000\\ 
         Valor Máximo&  12.8600&    0.3257&    0.1310\\ 
         Valor RMS&      4.7256&    0.1178&    0.0071\\
        \hline
    \end{tabular} \label{tb:comparacao_controle_UIO}\caption{Características do sistema de controle de vibração com observador de entradas desconhecidas}
\end{table}
\FloatBarrier

A análise dos resultados exibidos nas tabelas \ref{tb:comparacao_controle_LUE} e \ref{tb:comparacao_controle_UIO} indica que o observador de entradas desconhecidas performou muito melhor do que o observador de Luenberger. Os valores médios de todas as variáveis de interesse, $V_c$ $A_c$ e $Jerk_c$, estão próximos de zero, como esperado. Os valores máximos das variáveis são menores quando se emprega o observador UIO na malha de controle, sendo aproximadamente 3 vezes menor para todas as variáveis. Comportamento semelhante pode ser observado para o valor RMS das variáveis. 
Uma hipótese explicativa para este melhor desempenho da malha de controle é que a eficiência da estratégia de controle seja proporcional ao erro de estimativa da variável de estado. Como o observador UIO alcança valores muito menores de erro máximo e RMS na estimativa dos estados do sistema em comparação com o observador de Luenberger, este alcança melhor desempenho no controle em malha fechada.
\subsection{Comparação do desempenho do controlador robusto com \( \mathcal{D}\)-estabilidade garantida para perturbação em degrau}
Escolheu-se avaliar a eficiência das estratégias de controle utilziando o observador UIO em conjunto com o controlador robusto com \( \mathcal{D}\)-estabilidade garantida para uma perturbação em degrau unitário. Para isto, serão avaliados as leituras das variáveis do sistema $Xc$ e $VC$ quando o sistema é perturbado por uma entrada em degrau unitário. Os resultados obtidos pelo sistema controlado são comparados com os resultados do sistema em malha aberta para avaliação se os parâmetros de desempenho temporal são atendidos. As simulações foram realizadas para múltiplos valores do parâmetro $\delta$, com $-100kg\leq \delta\leq 100kg$ em passos de 5kg. O resultado da simulação temporal e exibido nas figuras \ref{fig:eval_cntrol_OPEN_step} e \ref{fig:eval_cntrol_UIO_step}  a seguir:

\FloatBarrier
\begin{figure}[htbp]
    \begin{centering}
    \includegraphics[width=8cm]{img/eval_cntrol_OPEN_step.png} 
    \caption{Resposta a perturbação em degrau do sistema em malha aberta}
    \label{fig:eval_cntrol_OPEN_step}
    \end{centering}
\end{figure}
\FloatBarrier

\FloatBarrier
\begin{figure}[htbp]
    \begin{centering}
    \includegraphics[width=8cm]{img/eval_cntrol_UIO_step.png} 
    \caption{Resposta a perturbação em degrau do sistema em malha fechada com o controlador robusto e o observador de entrada desconhecida}
    \label{fig:eval_cntrol_UIO_step}
    \end{centering}
\end{figure}
\FloatBarrier

Para o sistema funcionando em malha aberta é possível observar nos gráficos da figura \ref{fig:eval_cntrol_OPEN_step} que a amplitude e duração das oscilações da posição vertical da massa principal, $Xc$, não são amortecidas para todos os valores de $\delta$. Para uma perturbação $Xr$ em degrau unitário aplicada em $t=1$, o que equivale ao carro atropelar um barranco de 1 metro de altura, obtemos valor máximo de oscilação de $1.75m$ com tempo de acomodação de $7s$. Observa-se que o valor final da variável $Xc$ é sempre idêntico ao valor final da perturbação unitária. Para a variável velocidade de deslocamento vertical da massa principal, $Vc$, obtém-se um valor de pico de $10m\cdot s^{-1}$ em alguns casos com tempo de acomodação igual ao de $Xc$.  
Para o sistema funcionando em malha aberta é possível observar nos gráficos da figura \ref{fig:eval_cntrol_UIO_step} que a amplitude e duração das oscilações da posição vertical da massa principal, $Xc$, são rapidamente amortecidas para todos os valores de $\delta$. O sistema atinge acomodação em $0.5s$ para todos os valores de $delta$. As oscilações do transitório não superam o valor final da variável de forma visível. O valor final da variável $Xc$ neste caso é $0.2$. Não é zero mas é menor do que para o sistema não controlado, o que indica que seria necessário uma correção integral para uma rejeição a perturbação perfeita. O que indica que a perturbação foi absorvida principalmente pela deformação do conjunto amortecedor e mola, transferindo menos movimento ao veiculo.
Para a variável velocidade de deslocamento vertical da massa principal, $Vc$, obtém-se um valor de pico de $5m\cdot s^{-1}$ em alguns casos com tempo de acomodação de $0.5s$.  
Os resultados indicam que o controlador robusto com \( \mathcal{D}\)-estabilidade garantida foi eficiente em encontrar um vetor de ganhos de realimentação de estados que garantisse a acomodação a perturbação desejada, com máximo overshoot percentual de $M_O \leq 10$ \% e tempo de acomodação para a faixa de tolerância de 2 \% $T_S \leq 0.5$ s. 