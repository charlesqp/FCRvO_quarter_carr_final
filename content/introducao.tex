Uma suspensão veicular tem a função de isolar os passageiros e o chassi de vibrações originadas das irregularidades da estrada, além disso, atuam para garantir a estabilidade do veiculo durante manobras de condução em pista.
Existem três tipos de suspensão veicular: suspensão passiva, ativa e semiativa. Esta classificação é dada de acordo com a presença e o tipo de controle utilizado para minimizar as vibrações transmitidas ao chassi.
Sistemas de controle passivo são os sistemas de suspensão convencionais, compostos por molas, amortecedores e pneus. O sistema de controle passivo só é capaz de atuar em uma banda de frequência restrita, limitando-se a utilização em sistemas com frequências fora desta banda. Este tipo de sistema proporciona uma viagem menos confortável e estável ao carro em comparação aos sistemas de amortecimento ativo e semiativo. As característica importante deste tipo de sistema é ser caracterizado por não empregar energia externa ao sistema, além de ser o mais barato dentre todos.
Um sistema de suspensão ativa é um sistema capaz de atuar em diferentes bandas de frequência, através da utilização de atuadores, sensores e sistemas eletrônicos de controle. A desvantagem está na elevada quantidade de energia que os componentes utilizados necessitam, demandando o uso de uma fonte de energia externa, isto implica em um produto final de custo mais elevado quando comparado com sistemas de suspensão passiva.
Um sistema de suspensão semiativa também apresenta funcionalidade em diversas bandas de frequência, porém, não possui a obrigatoriedade de uma fonte de tensão externa permanente de grande porte. Outra vantagem deste tipo de sistema é que este, na falta de energia, comporta-se como um sistema passivo, agregando mais confiabilidade e segurança ao veículo. Por fim, um sistema de suspensão semiativa possui custo intermediário entre as demais opções.
O trabalho de \cite{Liu2019GeneralDesign} define a estratégia de controle  de suspensão Skyhook como uma suspensão semi-ativa que de fácil implementação com pouca informação sobre o estado do veículo. Neste tipo de controle, se destaca o trabalho de \cite{Karnopp1995ActiveAS} onde se estabelece um amortecedor virtual entre a carroceria do veículo e um ponto de referência de deslocamento vertical que fica no infinito, em um céu imaginário. É eficaz para aumentar o conforto de condução do veículo, mas a carga dinâmica do pneu deteriora-se ao mesmo tempo. 
Este trabalho aplica os conceitos de controle robusto otimizado via LMIs no controle de um atuador generalizado aplicado a estratégia Skyhook como alternativa as estratégias mais usuais. Além disso aborda uma forma diferenciada de controle robusto, predefinindo os parâmetros de resposta no tempo e sua robustez, garantindo as especificações sob condições de incerteza paramétrica do modelo e sua estabilidade robusta.