\subsection{Métodos de Lyapunov}
A bibliografia de engenharia de controle mostra o uso dos métodos de Lyapunov para a análise e síntese de controladores para sistemas lineares, como pode ser encontrado em \cite{ChenLSTI}. De uma forma geral, utiliza-se controladores lineares baseados nas funções de Lyapunov de forma quadrática. O controlador linear é baseado em modelos linearizados do sistema na proximidade de um ponto de equilíbrio, garantindo a estabilidade local do sistema controlado. Além disso, utilizando controladores é possível garantir a estabilidade do sistema em malha fechada, podendo se associar a qualquer controlador critérios de desempenho, como tempo de acomodação e máximo sobressinal (\cite{Hang1987RefinementsOT}; \cite{601347}; \cite{1049598}), normas $H_2$ (\cite{4789992}) e $H_\infty$ (\cite{Petersen}; \cite{Wang1992RobustCO}), taxa de convergência (\cite{Elia2001StabilizationInformation}; \cite{LORIA200213}), dentre outros.
\subsection{Problema de amortecimento de vibrações em veículos}
Uma suspensão veicular tem a função de isolar os passageiros e o chassi de vibrações originadas das irregularidades da estrada, além disso, atuam para garantir a estabilidade do veiculo durante manobras de condução em pista.
Existem três tipos de suspensão veicular: suspensão passiva, ativa e semiativa. Esta classificação é dada de acordo com a presença e o tipo de controle utilizado para minimizar as vibrações transmitidas ao chassi.
Sistemas de controle passivo são os sistemas de suspensão convencionais, compostos por molas, amortecedores e pneus. O sistema de controle passivo só é capaz de atuar em uma banda de frequência restrita, limitando-se a utilização em sistemas com frequências fora desta banda. Este tipo de sistema proporciona uma viagem menos confortável e estável ao carro em comparação aos sistemas de amortecimento ativo e semiativo. As característica importante deste tipo de sistema é ser caracterizado por não empregar energia externa ao sistema, além de ser o mais barato dentre todos.
Um sistema de suspensão ativa é um sistema capaz de atuar em diferentes bandas de frequência, através da utilização de atuadores, sensores e sistemas eletrônicos de controle. A desvantagem está na elevada quantidade de energia que os componentes utilizados necessitam, demandando o uso de uma fonte de energia externa, isto implica em um produto final de custo mais elevado quando comparado com sistemas de suspensão passiva.
Um sistema de suspensão semiativa também apresenta funcionalidade em diversas bandas de frequência, porém, não possui a obrigatoriedade de uma fonte de tensão externa permanente de grande porte. Outra vantagem deste tipo de sistema é que este, na falta de energia, comporta-se como um sistema passivo, agregando mais confiabilidade e segurança ao veículo. Por fim, um sistema de suspensão semiativa possui custo intermediário entre as demais opções.
O foco deste trabalho recai sobre o sistema de suspensão ativa com um atuador generalizado, apresentando assim a implementação de um controlador ativo aplicado a um sistema de suspensão veicular com modelo não linear.
\subsection{Observadores de estados}
Segundo a definição de \cite{EllisObserver}, um observador é uma estrutura matemática que, a cada instante de tempo t, constrói uma estimativa $\hat{x}(t)$ do estado $x(t)$ através da combinação das medidas das entradas $u(t)$ e saídas $y(t)$, provenientes de sensores. Em aplicações de controle de sistemas, mais especificamente no controle por realimentação de estados, a realimentação envolve a medição de todo o vetor de estado, o que nem sempre é possível ou viável economicamente. A solução neste caso é estimar os estados faltantes a partir das saídas que são possíveis de ser medidas. Desta forma, utiliza-se o vetor de ganhos calculado como se o estado fosse, de fato, medido. Assim, substitui-se o estado pelo estado estimado, multiplicado pelo vetor de ganhos do controlador, para fechar a malha de controle.
O estimador de estados foi chamado de observador por Luenberger em \cite{Luenberger1971AnObservers} e \cite{Luenberger}, o primeiro a apresentar o conceito e por este motivo é chamado de observador de Luenberger. Sua principal característica que o diferencia dos demais é que observador de Luenberger corrige a equação de estimação do estado com uma realimentação do erro de estimação $y(k)-\hat{y}(k)$.
De maneira geral, um problema dos observadores é o fato que perturbações externas corrompem o sinal do estado estimado, pois estas afetam a variável de estado real na planta controlada mas não são incluídas no calculo do preditor. De fato, perturbações não podem ser incluídas no cálculo da predição por que estas são desconhecidas na maioria das vezes. Com poucas excessões, perturbações não são medidas diretamente.  
\subsection{Observadores de Entradas Desconhecidas}
Observadores de Entradas Desconhecidas (do inglês,Unknown Input Observers, UIOs) são uma classe de observadores que possui como característica a possibilidade de desacoplar as entradas desconhecidas (erros de modelagem,distúrbios, perturbações, dentre outros) do erro de estimação. Um UIO trata o problema da corrupção do estado estimado, podendo ser aplicados para o projeto de controladores baseados em observadores (\cite{Zasadzinski1995LoopTR}), sendo que o UIO sintetizado é empregado para estimar os estados do sistema, os quais são posteriormente utilizados para realizar o controle por realimentação de estados. Outra aplicação para controle existente na literatura surge do uso de UIOs para estimativa de entradas desconhecidas (distúrbios)utilizada em aplicações como Disturbance Accommodation Control(DAC) (\cite{Chen2016Disturbance-Observer-BasedOverview}).