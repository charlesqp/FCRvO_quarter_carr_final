    Comparar o desempenho entre dois sistemas de controle empregando dois tipos diferentes de observador de estados. Em ambos os sistemas de controle será empregado o mesmo controlador de realimentação de estados, que será sintetizado por meio da técnica de \( \mathcal{D}\)-alocação de polos via desigualdades matriciais lineares (LMIs). 
    Para alcançar o objetivo principal, os seguintes objetivos específicos foram propostos:
    \begin{itemize}
        \item Realizar a modelagem matemática de um sistema de suspensão não linear de um quarto de veículo com dois graus de liberdade.
        \item Realizar a linearização do modelo completo proposto.
        \item Projetar um observador de Luemberger para a estimação das variáveis de estados necessárias para o funcionamento da estratégia de controle.
        \item Projetar um controlador linear por realimentação de estados com ganho constante por meio da aplicação da técnica de \( \mathcal{D}\)-alocação de polos via desigualdades matriciais lineares (LMIs). Uma \( \mathcal{D}\)-Região é proposta para garantir a estabilidade e os requisitos de desempenho de tempo de acomodação e máxima sobrelevação percentual mesmo sob condições de incerteza dos parâmetros do sistema.
        \item Projetar um observador de entradas desconhecidas, pelo método das desigualdades matriciais lineares (LMIs), para a estimação das variáveis de estados necessárias para o funcionamento da estratégia de controle.
        \item Realizar simulações temporais com o controlador proposto para o sistema linearizado empregando as diferentes metodologias de observadores e comparar os resultados.
    \end{itemize}